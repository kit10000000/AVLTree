\href{https://travis-ci.org/kit10000000/BSTTree}{\tt }

Целью данного проекта является разработка программы для работы с двоичным деревом поиска (далее будем называть просто деревом) со следующим T\+UI\+:

Выберите одну из операций\+:
\begin{DoxyEnumerate}
\item Вывести дерево на экран
\item Вывести список узлов дерева a. Прямой обход b. Поперечный обход c. Обратный обход
\item Добавить узел в дерево
\item Удалить узел из дерева
\item Сохранить дерево в файл
\item Загрузить дерево из файла
\item Проверить наличие узла
\item Завершить работу программы
\end{DoxyEnumerate}

Этапы\+:
\begin{DoxyEnumerate}
\item \+:deciduous\+\_\+tree\+:
\item \+:deciduous\+\_\+tree\+:
\item \+:deciduous\+\_\+tree\+:
\item \+:deciduous\+\_\+tree\+:
\item \+:deciduous\+\_\+tree\+:
\item \+:deciduous\+\_\+tree\+:
\item \+:deciduous\+\_\+tree\+:
\item \+:deciduous\+\_\+tree\+:
\end{DoxyEnumerate}

Требования\+:
\begin{DoxyEnumerate}
\item \+:white\+\_\+check\+\_\+mark\+: Все функции по работе с деревом должны находиться в пространстве имен \$\{Type\}Tree.
\item \+:white\+\_\+check\+\_\+mark\+: Структура узла должна иметь определённый прототип.
\item \+:white\+\_\+check\+\_\+mark\+: Класс дерева должна иметь определённый прототип.
\item \+:white\+\_\+check\+\_\+mark\+: Файлы исходного кода должны располагаться на сервисе Git\+Hub в открытом репозитории с названием \$\{Type\}Tree.
\item \+:white\+\_\+check\+\_\+mark\+: Репозиторий \$\{Type\}Tree должен иметь определённую структуру
\item \+:white\+\_\+check\+\_\+mark\+: Код должен быть читабельным и оформлен в едином стиле. astyle
\item \+:white\+\_\+check\+\_\+mark\+: Файлы исходного кода должны пройти проверку утилитой cpplint.
\item \+:white\+\_\+check\+\_\+mark\+: Оформить \mbox{\hyperlink{_r_e_a_d_m_e_8md}{R\+E\+A\+D\+M\+E.\+md}} файл, содержащий описание проекта.
\item \+:white\+\_\+check\+\_\+mark\+: Оформить .gitignore файл, создать отдельную ветку develop, каждый этап помечать соответствующим тэгом.
\item \+:white\+\_\+check\+\_\+mark\+: Создать C\+Make\+Lists.\+txt для автоматизации сборки проекта.
\item \+:white\+\_\+check\+\_\+mark\+: Добавить в C\+Make\+Lists.\+txt автоматизацию процесса сборки примеров.
\item \+:white\+\_\+check\+\_\+mark\+: Обеспечить 100\% покрытие кода с использованием фреймворка \mbox{\hyperlink{namespace_catch}{Catch}}.
\item \+:white\+\_\+check\+\_\+mark\+: Обеспечить непрерывный процесс сборки проекта с использованием сервиса Travis CI .
\item \+:x\+: Обеспечить непрерывный процесс сборки и тестирование проекта с использованием сервисов Travis CI и App\+Veyor.
\item \+:x\+: Написать документацию к проекту с использованием инструмента doxygen и разместить ее на сервисе Git\+Hub Page.
\item \+:white\+\_\+check\+\_\+mark\+: Добавить в C\+Make\+Lists.\+txt автоматизацию процесса пакетирования.
\item \+:x\+: Обеспечить размещение пакета проекта на сервисе Git\+Hub Release при успешном слиянии ветки develop и master.
\end{DoxyEnumerate}


\begin{DoxyCode}
$ git clone https://github.com/kit10000000/BSTTree.git BSTTree
$ cd BSTTree
$ git checkout develop
$ cmake -H. -B\_builds
$ cmake --build \_builds
\end{DoxyCode}
 